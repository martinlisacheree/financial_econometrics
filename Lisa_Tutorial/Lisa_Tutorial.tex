<<<<<<< HEAD
\documentclass[11pt,preprint, authoryear]{elsarticle}

\usepackage{lmodern}
%%%% My spacing
\usepackage{setspace}
\setstretch{1.2}
\DeclareMathSizes{12}{14}{10}{10}

% Wrap around which gives all figures included the [H] command, or places it "here". This can be tedious to code in Rmarkdown.
\usepackage{float}
\let\origfigure\figure
\let\endorigfigure\endfigure
\renewenvironment{figure}[1][2] {
    \expandafter\origfigure\expandafter[H]
} {
    \endorigfigure
}

\let\origtable\table
\let\endorigtable\endtable
\renewenvironment{table}[1][2] {
    \expandafter\origtable\expandafter[H]
} {
    \endorigtable
}


\usepackage{ifxetex,ifluatex}
\usepackage{fixltx2e} % provides \textsubscript
\ifnum 0\ifxetex 1\fi\ifluatex 1\fi=0 % if pdftex
  \usepackage[T1]{fontenc}
  \usepackage[utf8]{inputenc}
\else % if luatex or xelatex
  \ifxetex
    \usepackage{mathspec}
    \usepackage{xltxtra,xunicode}
  \else
    \usepackage{fontspec}
  \fi
  \defaultfontfeatures{Mapping=tex-text,Scale=MatchLowercase}
  \newcommand{\euro}{€}
\fi

\usepackage{amssymb, amsmath, amsthm, amsfonts}

\def\bibsection{\section*{References}} %%% Make "References" appear before bibliography


\usepackage[round]{natbib}
\bibliographystyle{plainnat}

\usepackage{longtable}
\usepackage[margin=2cm,bottom=2cm,top=2.5cm, includefoot]{geometry}
\usepackage{fancyhdr}
\usepackage[bottom, hang, flushmargin]{footmisc}
\usepackage{graphicx}
\numberwithin{equation}{section}
\numberwithin{figure}{section}
\numberwithin{table}{section}
\setlength{\parindent}{0cm}
\setlength{\parskip}{1.3ex plus 0.5ex minus 0.3ex}
\usepackage{textcomp}
\renewcommand{\headrulewidth}{0.2pt}
\renewcommand{\footrulewidth}{0.3pt}

\usepackage{array}
\newcolumntype{x}[1]{>{\centering\arraybackslash\hspace{0pt}}p{#1}}

%%%%  Remove the "preprint submitted to" part. Don't worry about this either, it just looks better without it:
\makeatletter
\def\ps@pprintTitle{%
  \let\@oddhead\@empty
  \let\@evenhead\@empty
  \let\@oddfoot\@empty
  \let\@evenfoot\@oddfoot
}
\makeatother

 \def\tightlist{} % This allows for subbullets!

\usepackage{hyperref}
\hypersetup{breaklinks=true,
            bookmarks=true,
            colorlinks=true,
            citecolor=blue,
            urlcolor=blue,
            linkcolor=blue,
            pdfborder={0 0 0}}


% The following packages allow huxtable to work:
\usepackage{siunitx}
\usepackage{multirow}
\usepackage{hhline}
\usepackage{calc}
\usepackage{tabularx}
\usepackage{booktabs}
\usepackage{caption}
\usepackage{colortbl}

\urlstyle{same}  % don't use monospace font for urls
\setlength{\parindent}{0pt}
\setlength{\parskip}{6pt plus 2pt minus 1pt}
\setlength{\emergencystretch}{3em}  % prevent overfull lines
\setcounter{secnumdepth}{0}

%%% Use protect on footnotes to avoid problems with footnotes in titles
\let\rmarkdownfootnote\footnote%
\def\footnote{\protect\rmarkdownfootnote}
\IfFileExists{upquote.sty}{\usepackage{upquote}}{}

%%% Include extra packages specified by user
% Insert custom packages here as follows
% \usepackage{tikz}

%%% Hard setting column skips for reports - this ensures greater consistency and control over the length settings in the document.
%% page layout
%% paragraphs
\setlength{\baselineskip}{12pt plus 0pt minus 0pt}
\setlength{\parskip}{12pt plus 0pt minus 0pt}
\setlength{\parindent}{0pt plus 0pt minus 0pt}
%% floats
\setlength{\floatsep}{12pt plus 0 pt minus 0pt}
\setlength{\textfloatsep}{20pt plus 0pt minus 0pt}
\setlength{\intextsep}{14pt plus 0pt minus 0pt}
\setlength{\dbltextfloatsep}{20pt plus 0pt minus 0pt}
\setlength{\dblfloatsep}{14pt plus 0pt minus 0pt}
%% maths
\setlength{\abovedisplayskip}{12pt plus 0pt minus 0pt}
\setlength{\belowdisplayskip}{12pt plus 0pt minus 0pt}
%% lists
\setlength{\topsep}{10pt plus 0pt minus 0pt}
\setlength{\partopsep}{3pt plus 0pt minus 0pt}
\setlength{\itemsep}{5pt plus 0pt minus 0pt}
\setlength{\labelsep}{8mm plus 0mm minus 0mm}
\setlength{\parsep}{\the\parskip}
\setlength{\listparindent}{\the\parindent}
%% verbatim
\setlength{\fboxsep}{5pt plus 0pt minus 0pt}



\begin{document}

\begin{frontmatter}  %

\title{Texevier Tutorial}

% Set to FALSE if wanting to remove title (for submission)




\author[Add1]{Lisa-Cheree Martin}
\ead{18190642@sun.ac.za}





\address[Add1]{Stellenbosch University, Stellenbosch, South Africa}

\cortext[cor]{Corresponding author: Lisa-Cheree Martin}

\begin{abstract}
\small{
This is an assignment for the Financial Econometrics 871 course offered
at Stellenbosch University. The aim of this tut is to work in Texevier
and get a sense of what is required for the larger research project. The
results are secondary to the goal of successfully being able to create a
neat write-up and include figures and tables with results.
}
\end{abstract}

\vspace{1cm}

\begin{keyword}
\footnotesize{
Univariate GARCH \\ \vspace{0.3cm}
\textit{JEL classification} 
}
\end{keyword}
\vspace{0.5cm}
\end{frontmatter}



%________________________
% Header and Footers
%%%%%%%%%%%%%%%%%%%%%%%%%%%%%%%%%
\pagestyle{fancy}
\chead{}
\rhead{Financial Econometrics 871}
\lfoot{}
\rfoot{\footnotesize Page \thepage\\}
\lhead{}
%\rfoot{\footnotesize Page \thepage\ } % "e.g. Page 2"
\cfoot{}

%\setlength\headheight{30pt}
%%%%%%%%%%%%%%%%%%%%%%%%%%%%%%%%%
%________________________

\headsep 35pt % So that header does not go over title




\section{\texorpdfstring{Question 1
\label{Q1}}{Question 1 }}\label{question-1}

\subsection{Create a summary table showing the first and second moments
of the returns of these stocks for the following
periods:}\label{create-a-summary-table-showing-the-first-and-second-moments-of-the-returns-of-these-stocks-for-the-following-periods}

\begin{itemize}
\tightlist
\item
  2006 - 2008
\end{itemize}

\begin{table}[H]
\centering
\scalebox{0.9}{
\begin{tabular}{rlrr}
  \hline
 & Stock & mean & variance \\ 
  \hline
1 & JSE.ABSP.Close & -0.02 & 0.01 \\ 
  2 & JSE.BVT.Close & 0.03 & 0.05 \\ 
  3 & JSE.FSR.Close & 0.01 & 0.06 \\ 
  4 & JSE.NBKP.Close & -0.03 & 0.01 \\ 
  5 & JSE.RMH.Close & 0.03 & 0.07 \\ 
  6 & JSE.SBK.Close & 0.04 & 0.06 \\ 
  7 & JSE.SLM.Close & 0.04 & 0.05 \\ 
   \hline
\end{tabular}
}
\caption{1st and 2nd Moments of Stock Returns, 2006-2008 \label{tab1}} 
\end{table}

\begin{itemize}
\tightlist
\item
  2010 - 2013
\end{itemize}

\begin{table}[H]
\centering
\scalebox{0.9}{
\begin{tabular}{rlrr}
  \hline
 & Stock & mean & variance \\ 
  \hline
1 & JSE.ABSP.Close & 0.01 & 0.01 \\ 
  2 & JSE.BVT.Close & 0.08 & 0.02 \\ 
  3 & JSE.FSR.Close & 0.08 & 0.03 \\ 
  4 & JSE.NBKP.Close & 0.01 & 0.01 \\ 
  5 & JSE.RMH.Close & 0.07 & 0.04 \\ 
  6 & JSE.SBK.Close & 0.03 & 0.02 \\ 
  7 & JSE.SLM.Close & 0.09 & 0.02 \\ 
   \hline
\end{tabular}
}
\caption{1st and 2nd Moments of Stock Returns, 2010-2013 \label{tab2}} 
\end{table}

For this question, first the returns of each stock over the entire
period were calculated. Next, the first and second moments of each are
calculated over the two periods before (and including) and after the GFC
of 2007/08. The resulting moments of each stock are then summarised
(above).

Note: The returns calculation produces two datasets: 1) a tidy dataset
with the returns of each stock, 2) a wide dataset with the log-returns
for each stock (for use at a later stage). The wide dataset is created
by using the \texttt{spread()} function on the tidy dataset, after
removing the \emph{Close} and \emph{Return} columns.

\subsection{Comment on the differences between the different
periods}\label{comment-on-the-differences-between-the-different-periods}

Mean returns in the period 2006-2008 are lower than in the period
2010-2013. This makes sense as the first period includes the Global
Financial Crisis. The variance is also lower during the period following
the crisis.The high variance during the first period is indicative of
the panic that struck during the crisis, and the high volatility that
followed the crash.

Look at the following completely random in-text reference: Tsay
(\protect\hyperlink{ref-tsay1989}{1989})

\section{\texorpdfstring{Question 2
\label{Q2}}{Question 2 }}\label{question-2}

\subsection{Calculate the unconditional (full sample) correlations
between the
stocks.}\label{calculate-the-unconditional-full-sample-correlations-between-the-stocks.}

\begin{table}[H]
\centering
\scalebox{0.7}{
\begin{tabular}{rrrrrrrr}
  \hline
 & JSE.ABSP.Close & JSE.BVT.Close & JSE.FSR.Close & JSE.NBKP.Close & JSE.RMH.Close & JSE.SBK.Close & JSE.SLM.Close \\ 
  \hline
JSE.ABSP.Close & 1.00 & -0.42 & -0.44 & 0.92 & -0.44 & -0.48 & -0.48 \\ 
  JSE.BVT.Close & -0.42 & 1.00 & 0.95 & -0.41 & 0.93 & 0.90 & 0.98 \\ 
  JSE.FSR.Close & -0.44 & 0.95 & 1.00 & -0.43 & 0.98 & 0.93 & 0.97 \\ 
  JSE.NBKP.Close & 0.92 & -0.41 & -0.43 & 1.00 & -0.43 & -0.45 & -0.48 \\ 
  JSE.RMH.Close & -0.44 & 0.93 & 0.98 & -0.43 & 1.00 & 0.94 & 0.94 \\ 
  JSE.SBK.Close & -0.48 & 0.90 & 0.93 & -0.45 & 0.94 & 1.00 & 0.91 \\ 
  JSE.SLM.Close & -0.48 & 0.98 & 0.97 & -0.48 & 0.94 & 0.91 & 1.00 \\ 
   \hline
\end{tabular}
}
\caption{Unconditional Correlations Between the Stocks \label{tab3}} 
\end{table}

\section{\texorpdfstring{Question 3
\label{Q3}}{Question 3 }}\label{question-3}

\subsection{Plot the univariate GARCH ht processes for each of the
series.}\label{plot-the-univariate-garch-ht-processes-for-each-of-the-series.}

The wide dataset of log-returns of each stock (produced in the returns
calculation of Question 1 \ref{Q1}) is used for this question.

\begin{verbatim}
## 
## please wait...calculating quantiles...
\end{verbatim}

\begin{figure}[H]

{\centering \includegraphics{Lisa_Tutorial_files/figure-latex/figure1-1} 

}

\caption{GARCH Univariate Plot \label{plot1}}\label{fig:figure1}
\end{figure}

\section{\texorpdfstring{Question 4
\label{Q4}}{Question 4 }}\label{question-4}

\subsection{Plot the cumulative returns series of a portfolio that is
equally weighted to each of the stocks - reweighted each year on the
last day of
June}\label{plot-the-cumulative-returns-series-of-a-portfolio-that-is-equally-weighted-to-each-of-the-stocks---reweighted-each-year-on-the-last-day-of-june}

\begin{figure}[H]

{\centering \includegraphics{Lisa_Tutorial_files/figure-latex/figure2-1} 

}

\caption{Cumulative Plot \label{plot2}}\label{fig:figure2}
\end{figure}

The above plot illusatrates the cumulative returns of the equally
weighted portfolio of the stocks, over the entire period. I was unable
to annually reweight the portfolio on the last day of June.

\newpage

\section*{References}\label{references}
\addcontentsline{toc}{section}{References}

\hypertarget{refs}{}
\hypertarget{ref-tsay1989}{}
Tsay, Ruey S. 1989. ``Testing and Modeling Threshold Autoregressive
Processes.'' \emph{Journal of the American Statistical Association} 84
(405). Taylor \& Francis Group: 231--40.

% Force include bibliography in my chosen format:

\bibliographystyle{Tex/Texevier}
\bibliography{Tex/ref}





\end{document}
=======
\documentclass[11pt,preprint, authoryear]{elsarticle}

\usepackage{lmodern}
%%%% My spacing
\usepackage{setspace}
\setstretch{1.2}
\DeclareMathSizes{12}{14}{10}{10}

% Wrap around which gives all figures included the [H] command, or places it "here". This can be tedious to code in Rmarkdown.
\usepackage{float}
\let\origfigure\figure
\let\endorigfigure\endfigure
\renewenvironment{figure}[1][2] {
    \expandafter\origfigure\expandafter[H]
} {
    \endorigfigure
}

\let\origtable\table
\let\endorigtable\endtable
\renewenvironment{table}[1][2] {
    \expandafter\origtable\expandafter[H]
} {
    \endorigtable
}


\usepackage{ifxetex,ifluatex}
\usepackage{fixltx2e} % provides \textsubscript
\ifnum 0\ifxetex 1\fi\ifluatex 1\fi=0 % if pdftex
  \usepackage[T1]{fontenc}
  \usepackage[utf8]{inputenc}
\else % if luatex or xelatex
  \ifxetex
    \usepackage{mathspec}
    \usepackage{xltxtra,xunicode}
  \else
    \usepackage{fontspec}
  \fi
  \defaultfontfeatures{Mapping=tex-text,Scale=MatchLowercase}
  \newcommand{\euro}{€}
\fi

\usepackage{amssymb, amsmath, amsthm, amsfonts}

\def\bibsection{\section*{References}} %%% Make "References" appear before bibliography


\usepackage[round]{natbib}
\bibliographystyle{plainnat}

\usepackage{longtable}
\usepackage[margin=2cm,bottom=2cm,top=2.5cm, includefoot]{geometry}
\usepackage{fancyhdr}
\usepackage[bottom, hang, flushmargin]{footmisc}
\usepackage{graphicx}
\numberwithin{equation}{section}
\numberwithin{figure}{section}
\numberwithin{table}{section}
\setlength{\parindent}{0cm}
\setlength{\parskip}{1.3ex plus 0.5ex minus 0.3ex}
\usepackage{textcomp}
\renewcommand{\headrulewidth}{0.2pt}
\renewcommand{\footrulewidth}{0.3pt}

\usepackage{array}
\newcolumntype{x}[1]{>{\centering\arraybackslash\hspace{0pt}}p{#1}}

%%%%  Remove the "preprint submitted to" part. Don't worry about this either, it just looks better without it:
\makeatletter
\def\ps@pprintTitle{%
  \let\@oddhead\@empty
  \let\@evenhead\@empty
  \let\@oddfoot\@empty
  \let\@evenfoot\@oddfoot
}
\makeatother

 \def\tightlist{} % This allows for subbullets!

\usepackage{hyperref}
\hypersetup{breaklinks=true,
            bookmarks=true,
            colorlinks=true,
            citecolor=blue,
            urlcolor=blue,
            linkcolor=blue,
            pdfborder={0 0 0}}


% The following packages allow huxtable to work:
\usepackage{siunitx}
\usepackage{multirow}
\usepackage{hhline}
\usepackage{calc}
\usepackage{tabularx}
\usepackage{booktabs}
\usepackage{caption}
\usepackage{colortbl}

\urlstyle{same}  % don't use monospace font for urls
\setlength{\parindent}{0pt}
\setlength{\parskip}{6pt plus 2pt minus 1pt}
\setlength{\emergencystretch}{3em}  % prevent overfull lines
\setcounter{secnumdepth}{0}

%%% Use protect on footnotes to avoid problems with footnotes in titles
\let\rmarkdownfootnote\footnote%
\def\footnote{\protect\rmarkdownfootnote}
\IfFileExists{upquote.sty}{\usepackage{upquote}}{}

%%% Include extra packages specified by user
% Insert custom packages here as follows
% \usepackage{tikz}

%%% Hard setting column skips for reports - this ensures greater consistency and control over the length settings in the document.
%% page layout
%% paragraphs
\setlength{\baselineskip}{12pt plus 0pt minus 0pt}
\setlength{\parskip}{12pt plus 0pt minus 0pt}
\setlength{\parindent}{0pt plus 0pt minus 0pt}
%% floats
\setlength{\floatsep}{12pt plus 0 pt minus 0pt}
\setlength{\textfloatsep}{20pt plus 0pt minus 0pt}
\setlength{\intextsep}{14pt plus 0pt minus 0pt}
\setlength{\dbltextfloatsep}{20pt plus 0pt minus 0pt}
\setlength{\dblfloatsep}{14pt plus 0pt minus 0pt}
%% maths
\setlength{\abovedisplayskip}{12pt plus 0pt minus 0pt}
\setlength{\belowdisplayskip}{12pt plus 0pt minus 0pt}
%% lists
\setlength{\topsep}{10pt plus 0pt minus 0pt}
\setlength{\partopsep}{3pt plus 0pt minus 0pt}
\setlength{\itemsep}{5pt plus 0pt minus 0pt}
\setlength{\labelsep}{8mm plus 0mm minus 0mm}
\setlength{\parsep}{\the\parskip}
\setlength{\listparindent}{\the\parindent}
%% verbatim
\setlength{\fboxsep}{5pt plus 0pt minus 0pt}



\begin{document}

\begin{frontmatter}  %

\title{Texevier Tutorial}

% Set to FALSE if wanting to remove title (for submission)




\author[Add1]{Lisa-Cheree Martin}
\ead{18190642@sun.ac.za}





\address[Add1]{Stellenbosch University, Stellenbosch, South Africa}

\cortext[cor]{Corresponding author: Lisa-Cheree Martin}

\begin{abstract}
\small{
This is an assignment for the Financial Econometrics 871 course offered
at Stellenbosch University. The aim of this tut is to work in Texevier
and get a sense of what is required for the larger research project. The
results are secondary to the goal of successfully being able to create a
neat write-up and include figures and tables with results.
}
\end{abstract}

\vspace{1cm}

\begin{keyword}
\footnotesize{
Univariate GARCH \\ \vspace{0.3cm}
\textit{JEL classification} 
}
\end{keyword}
\vspace{0.5cm}
\end{frontmatter}



%________________________
% Header and Footers
%%%%%%%%%%%%%%%%%%%%%%%%%%%%%%%%%
\pagestyle{fancy}
\chead{}
\rhead{Financial Econometrics 871}
\lfoot{}
\rfoot{\footnotesize Page \thepage\\}
\lhead{}
%\rfoot{\footnotesize Page \thepage\ } % "e.g. Page 2"
\cfoot{}

%\setlength\headheight{30pt}
%%%%%%%%%%%%%%%%%%%%%%%%%%%%%%%%%
%________________________

\headsep 35pt % So that header does not go over title




\section{\texorpdfstring{Question 1
\label{Q1}}{Question 1 }}\label{question-1}

\subsection{Create a summary table showing the first and second moments
of the returns of these stocks for the following
periods:}\label{create-a-summary-table-showing-the-first-and-second-moments-of-the-returns-of-these-stocks-for-the-following-periods}

\begin{itemize}
\tightlist
\item
  2006 - 2008
\end{itemize}

\begin{table}[H]
\centering
\scalebox{0.9}{
\begin{tabular}{rlrr}
  \hline
 & Stock & mean & variance \\ 
  \hline
1 & JSE.ABSP.Close & -0.02 & 0.01 \\ 
  2 & JSE.BVT.Close & 0.03 & 0.05 \\ 
  3 & JSE.FSR.Close & 0.01 & 0.06 \\ 
  4 & JSE.NBKP.Close & -0.03 & 0.01 \\ 
  5 & JSE.RMH.Close & 0.03 & 0.07 \\ 
  6 & JSE.SBK.Close & 0.04 & 0.06 \\ 
  7 & JSE.SLM.Close & 0.04 & 0.05 \\ 
   \hline
\end{tabular}
}
\caption{1st and 2nd Moments of Stock Returns, 2006-2008 \label{tab1}} 
\end{table}

\begin{itemize}
\tightlist
\item
  2010 - 2013
\end{itemize}

\begin{table}[H]
\centering
\scalebox{0.9}{
\begin{tabular}{rlrr}
  \hline
 & Stock & mean & variance \\ 
  \hline
1 & JSE.ABSP.Close & 0.01 & 0.01 \\ 
  2 & JSE.BVT.Close & 0.08 & 0.02 \\ 
  3 & JSE.FSR.Close & 0.08 & 0.03 \\ 
  4 & JSE.NBKP.Close & 0.01 & 0.01 \\ 
  5 & JSE.RMH.Close & 0.07 & 0.04 \\ 
  6 & JSE.SBK.Close & 0.03 & 0.02 \\ 
  7 & JSE.SLM.Close & 0.09 & 0.02 \\ 
   \hline
\end{tabular}
}
\caption{1st and 2nd Moments of Stock Returns, 2010-2013 \label{tab2}} 
\end{table}

For this question, first the returns of each stock over the entire
period were calculated. Next, the first and second moments of each are
calculated over the two periods before (and including) and after the GFC
of 2007/08. The resulting moments of each stock are then summarised
(above).

Note: The returns calculation produces two datasets: 1) a tidy dataset
with the returns of each stock, 2) a wide dataset with the log-returns
for each stock (for use at a later stage). The wide dataset is created
by using the \texttt{spread()} function on the tidy dataset, after
removing the \emph{Close} and \emph{Return} columns.

\subsection{Comment on the differences between the different
periods}\label{comment-on-the-differences-between-the-different-periods}

Mean returns in the period 2006-2008 are lower than in the period
2010-2013. This makes sense as the first period includes the Global
Financial Crisis. The variance is also lower during the period following
the crisis.The high variance during the first period is indicative of
the panic that struck during the crisis, and the high volatility that
followed the crash.

Look at the following completely random in-text reference: Tsay
(\protect\hyperlink{ref-tsay1989}{1989})

\section{\texorpdfstring{Question 2
\label{Q2}}{Question 2 }}\label{question-2}

\subsection{Calculate the unconditional (full sample) correlations
between the
stocks.}\label{calculate-the-unconditional-full-sample-correlations-between-the-stocks.}

\begin{table}[H]
\centering
\scalebox{0.7}{
\begin{tabular}{rrrrrrrr}
  \hline
 & JSE.ABSP.Close & JSE.BVT.Close & JSE.FSR.Close & JSE.NBKP.Close & JSE.RMH.Close & JSE.SBK.Close & JSE.SLM.Close \\ 
  \hline
JSE.ABSP.Close & 1.00 & -0.42 & -0.44 & 0.92 & -0.44 & -0.48 & -0.48 \\ 
  JSE.BVT.Close & -0.42 & 1.00 & 0.95 & -0.41 & 0.93 & 0.90 & 0.98 \\ 
  JSE.FSR.Close & -0.44 & 0.95 & 1.00 & -0.43 & 0.98 & 0.93 & 0.97 \\ 
  JSE.NBKP.Close & 0.92 & -0.41 & -0.43 & 1.00 & -0.43 & -0.45 & -0.48 \\ 
  JSE.RMH.Close & -0.44 & 0.93 & 0.98 & -0.43 & 1.00 & 0.94 & 0.94 \\ 
  JSE.SBK.Close & -0.48 & 0.90 & 0.93 & -0.45 & 0.94 & 1.00 & 0.91 \\ 
  JSE.SLM.Close & -0.48 & 0.98 & 0.97 & -0.48 & 0.94 & 0.91 & 1.00 \\ 
   \hline
\end{tabular}
}
\caption{Unconditional Correlations Between the Stocks \label{tab3}} 
\end{table}

\section{\texorpdfstring{Question 3
\label{Q3}}{Question 3 }}\label{question-3}

\subsection{Plot the univariate GARCH ht processes for each of the
series.}\label{plot-the-univariate-garch-ht-processes-for-each-of-the-series.}

The wide dataset of log-returns of each stock (produced in the returns
calculation of Question 1 \ref{Q1}) is used for this question.

\begin{verbatim}
## 
## please wait...calculating quantiles...
\end{verbatim}

\begin{figure}[H]

{\centering \includegraphics{Lisa_Tutorial_files/figure-latex/figure1-1} 

}

\caption{GARCH Univariate Plot \label{plot1}}\label{fig:figure1}
\end{figure}

\section{\texorpdfstring{Question 4
\label{Q4}}{Question 4 }}\label{question-4}

\subsection{Plot the cumulative returns series of a portfolio that is
equally weighted to each of the stocks - reweighted each year on the
last day of
June}\label{plot-the-cumulative-returns-series-of-a-portfolio-that-is-equally-weighted-to-each-of-the-stocks---reweighted-each-year-on-the-last-day-of-june}

\begin{figure}[H]

{\centering \includegraphics{Lisa_Tutorial_files/figure-latex/figure2-1} 

}

\caption{Cumulative Plot \label{plot2}}\label{fig:figure2}
\end{figure}

The above plot illusatrates the cumulative returns of the equally
weighted portfolio of the stocks, over the entire period. I was unable
to annually reweight the portfolio on the last day of June.

\newpage

\section*{References}\label{references}
\addcontentsline{toc}{section}{References}

\hypertarget{refs}{}
\hypertarget{ref-tsay1989}{}
Tsay, Ruey S. 1989. ``Testing and Modeling Threshold Autoregressive
Processes.'' \emph{Journal of the American Statistical Association} 84
(405). Taylor \& Francis Group: 231--40.

% Force include bibliography in my chosen format:

\bibliographystyle{Tex/Texevier}
\bibliography{Tex/ref}





\end{document}
>>>>>>> 3dc00ed77eacc1a7de76b869918382bec92f4710
